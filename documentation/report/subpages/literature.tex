\chapter{Literature Review}

\section{Background and Recent Work}

\subsection{Real-Time Emotion Detection using RPi2}
- By Suja, P. et al. \cite{suja2016} \\
The methodology in this paper involved the use of Raspberry Pi II and CMU MultiPIE database to detect emotions in real-time. The advantages of this methology was that the execution was done at real-time, and had a high speed and high accuracy. However, it only used the features extracted from facial expressions in the image (Speech was considered under `future work').

\subsection{Image Processing on RPi in Matlab}
- Horak, K. et al. \cite{horak2015} \\
The methodology in this paper involved the use of Raspberry Pi II and Simulink in Matlab to process images with many filters such as the Sobel filter. The advantages of this methology was that the execution was done at real-time, and had a high speed. It also extracted features based on edge, corner, line detection. However, it lead to increase in FPS due to transfer from RPi2 to Simulink.

\subsection{Real-Time Face Recognition using RPi2}
- Viji, A. et al. \cite{viji2017} \\
The methodology in this paper involved the use of Raspberry Pi II, Haar cascade classifier, PCA feature extraction, and Adaboost classification to detect faces in real-time. The advantages of this methology was that the execution was done at real-time, and had a high speed and high accuracy. However, it considers only PCA feature extraction.

\pagebreak

\subsection{Robust Real-Time Face Detection}
- Viola, P. et al. \cite{viola2004} \\
The methodology in this paper involved the use of Haar feature selection to create an internal image, and Adaboost training and Cascading classifiers to detect faces in real-time. The advantages of this methology was that the execution required minimal computation time and had a high detection accuracy. However, real-time conditions (illuminations, non-uniform conditions) were ignored.

\section{Outcomes of Literature Review}
Current state-of-art implementations consider facial expressions to be the primary source of emotion detection. Various methods for feature extraction have been used in the past like ASM, PCA which employ geometric-based feature detection. Processing on Raspberry Pi can give speeds upto 100ms and Viola-Jones feature detection gives about 95\% accuracy. Current state-of-art implementations do not consider parallelization of the process. 

\section{Problem Statement}
Face detection, feature extraction, ad finally classification of an image to a particular recognizable emotion. Also implementing the same using a parallel (master-slave) model using Quad-Pi (four RPi3).

\section{Research Objectives}
\begin{itemize}
\item To develop emotion recognition from facial images to simulate and experiment in near real-time environment, programming the Raspberry Pi III, deals with Image Processing (cropping and face extraction), feature extraction from grayscale face image extraction, and classify using Adaptive Boosting classifier trained with CMU MultiPIE database
\item Use Quad-Pi (four RPi3) to implement master-slave parallel processing model to the deployed application.
\end{itemize}
